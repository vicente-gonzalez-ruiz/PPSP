\title{\href{https://datatracker.ietf.org/wg/ppsp}{PPSP (Peer to Peer Streaming Protocol)}}
\author{Vicente González Ruiz}

\maketitle
\tableofcontents

\section{What is the PPSP?}
%{{{

\begin{itemize}
\item The PPSP is an application-layer protocol for the real-time
  streaming of media content between (usually networked) processes
  (peers), as well as the delivery of content-on-demand.
\item The PPSP comprises two different protocols:
  \begin{enumerate}
  \item The \textbf{PPSP-TP (PPSP Tracker Protocol):} The tracker
    protocol handles the initial and periodic exchange of meta
    information between trackers and peers, such as which peers can
    provide the content. A descentralized tracking mechanism has also
    been developed.
  \item The \textbf{PPSPP (PPS Peer Protocol):} The peer protocol
    controls the exchange of the content between the peers.
  \end{enumerate}
\end{itemize}

%}}}

\section{Basic operation}
%{{{

\begin{itemize}
\item When a peer wants to receive streaming of a selected content (leech
   mode):
\begin{enumerate}
\item Peer connects to a tracker and joins a swarm.
\item Peer acquires a list of other peers in the swarm from the tracker.
\item Peer exchanges its content availability with the peers on the
  obtained peer list (peer protocol).
\item Peer identifies the peers with desired content.
\item Peer requests content from the identified peers (peer protocol).
\end{enumerate}
\end{itemize}

%}}}

\section{Transport protocols}
%{{{

\begin{itemize}
\item Usually, the UDP is used for transport in the PPSPP. In this
  case, each chunk is explicity acknowledged.
\item The PPSP can use also the TCP.
\end{itemize}

%}}}

\section{Architecture}
%{{{

\begin{itemize}
\item The content is provided by seeder peers. Several seeder peers
  can be in the same swarm (content-on-demand).
\item The peers meet each other asking a tracker (the tracker does not
  handle the content) or an other (connected) peers.
\item Peers are ``connected''\footnote{Using the UDP the connection
    concept does not make sense.} ramdomly, depending on the churn and
  the network behaviour.
\end{itemize}

%}}}

\section{Chunk delivery}
%{{{

\begin{itemize}
\item Peers annouce to the peers that are connected the chunks that it
  has using ``HAVE'' messages. In general, HAVE messages are
  piggybacked onto DATA (chunk) messages.
\item Peers request chunks using ``REQUEST'' messges (pull scheme).
\item Peers transmitt as much content as possible, depending on the
  requests of the others peers, although a peer can ``choke'' (avoid
  to send data) other peer for some reason (for example, because the
  upload capacity is exhausted).
\end{itemize}

%}}}

\section{Freeriding prevention}
%{{{

\begin{itemize}
\item Freeriding (how to deal with peers that only download content
  byt never upload) is optional (it is defined as an extension).
\item Proposes to use the Tit-For-Tat Algorithm as used in BitTorrent
  (initially, a peer A tries to upload as much as possible to another
  peer B and if B does not recriprocate, A chokes B and try with a
  different randomly selected peer).
\end{itemize}

%}}}

\section{Content integrity}
%{{{

\begin{itemize}
\item Peers receive a cryptographic hash for earch chunk of data which
  can be used to check if the content has been modified in transit and
  allows every peer to directly detect when a malicious peer tries to
  distribute fake content.
\item When a peer detects a malicious peer, it is placed on a
  blacklist and not served anymore.
\end{itemize}

%}}}

\section{Swarms (channels)}
%{{{

\begin{itemize}
\item Peers can listen to multiple swarms when they desire to receive
  several channels.
\end{itemize}

%}}}

